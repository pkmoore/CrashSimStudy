\begin{abstract}

A common problem for software developers is that applications tend to
exhibit new bugs \textit{after} they have been deployed.
Preventing bugs related to
network,
operating system,
file system,
and other similar environmental features
has proven difficult in the past
largely because there is no way to
know how an application will react to every environment without
time-consuming and expensive testing.
Enter CrashSimulator, a tool that utilizes
evidence of an application's failure
in a given environment
as a predictor of
potential inadequacies
in others.
The key to the tool
is a technique called
\textit{Anomalous Environment Simulation},
which can extract
unusual environmental properties
found in the interactions between an application
and its environment
and use them to
recreate the environment in question.
Simulating these features
on a running application
imposes anomalous environmental conditions
on it.
If the application does not respond correctly to a given scenario,
the tool reports that a bug may exist.
These anomalies can then be stored in the tool,
so,
over time,
a developer will be able
to test
new applications
for an increasingly wider range
of potential bugs
before deployment.
To test the tool,
we evaluated CrashSimulator against a set of the most popular
Linux applications selected
from the Coreutils project and the Debian popularity contest.
Our tests found a total of XXXX bugs in YYYY applications,
the consequences of which range from hangs and crashes, to data loss and
remote denial of service conditions.

\end{abstract}
