\begin{abstract}

    % What CrashSimulator Does vs. What This Paper Does
    The proliferation software applications available today, coupled with a
    shortage of developer time , necessitates the use of automated testing
    tools to find and eliminate bugs prior to deployment.
    Developers can choose from a
    variety of tools depending on the desired degrees of complexity
    and/or their bug-finding capabilities.  However, adopting a new tool
    can result in a great deal of wasted effort if its learning curve
    outweighs the value of its results.  In this work we introduce
    CrashSimulator, a tool that identifies bugs created by anomalous
    conditions in its intended \textit{before} deployment.
    executed.  In this way, CrashSimulator allows developers to find
    environmental bugs regardless of the amount experience they may have
    with the underlying operating systems concepts.

    In order to demonstrate this claim, we directly compare CrashSimulator
    to two similar tools (AFL and Mutiny) in a study with ZZZ users.
    Participants were asked to use these tools to hunt for bugs
    in popular, real-world applications.  This study revealed that
    CrashSimulator's strategy of identifying bugs visible in system call
    sequences is effective, and that the tool can be used by developers
    with diverse backgrounds, and across skill sets.  Though developers
    with a high degree of proficiency with operating systems concepts
    appreciated CrashSimulator's ability to target unusual environmental
    conditions, even subjects
    had little experience did not appear to be overwhelmed by the tool's
    setup and operation.
    As YYY novel bugs were identified and reported during the study,
    even when used by novice
    developers, CrashSimulator is poised to help uncover
    a great deal of previously unidentified bugs in new and existing
    codebases.

\end{abstract}
