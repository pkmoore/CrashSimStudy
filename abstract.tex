\begin{abstract}

    % What CrashSimulator Does vs. What This Paper Does
    The sheer scale of software applications available today, as well as a
    shortage of developer time, necessitates the use of automated testing
    tools to find and eliminate bugs.  Today's developers can choose from a
    huge variety of tools that offer varying degrees of complexity and
    bug-finding capabilities.  However, adopting a new tool can result in a
    great deal of wasted effort if its learning curve outweighs the value
    of its results.  In this work we introduce CrashSimulator -- a tool
    that identifies bugs created by anomalous conditions present in the
    environment in which an application is executed.  CrashSimulator finds
    these bugs by recording, modifying, and replaying the results and side
    effects of the system calls made by the application. Moreover, unlike
    most otehr toosl, CrashSimulator can be used even by novice developers
    with little experience.  In order to demonstrate that CrashSimulator
    can assist by developers regardless of skill level, this work compares
    it to two similar tools (AFL and Mutiny).  To facilitate this
    comparison we conducted a study with ZZZ users in which participants
    were asked to use the tools in question to hunt for bugs in popular,
    real-world applications.  This study revealed that CrashSimulator's
    strategy of identifying bugs visible in system call sequences is
    effective and usable for developers with diverse backgrounds, and
    across skill sets like years of programming experience.  Developers
    with a high degree of proficiency with operating systems concepts
    preferred CrashSimulator's ability to target unusual environmental
    conditions over manipulating applicaiton data inputs.  Our participants
    reported that CrashSimulator's additional setup complexity was
    dramatically outweighed by its ability to find new bugs.  During the
    study YYY novel bugs were identified and reported, ZZZ of which have
    been since been corrected.  With wider use CrashSimulator could help
    uncover a great deal of previously unidentified bugs in new and
    existing codebases.

\end{abstract}
