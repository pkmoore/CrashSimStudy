\begin{abstract}
% What CrashSimulator Does vs. What This Paper Does
The sheer scale of software applications in use today, as well as a shortage of
developer time, necessitate usage of automated testing tools in combating bugs.
Today's developers can choose from a huge variety of tools with varying degrees
of complexity and bug-finding capability.  However, adopting a new tool can
result in a great deal of wasted effort if its learning curve outweighs the
value of its results.  In this work we introduce CrashSimulator -- a tool that
identifies bugs resultant from unhandled an anomalous conditions present in the
environment in which an application is executing.  CrashSimulator does this by
recording, modifying, and replaying the results and side effects of the system
calls an applications makes.  In order to demonstrate that CrashSimulator is
worthy consideration by developers regardless of skill level this work compares
it to two similar tools (AFL and Mutiny) in a study of developer's ability to
use the tools in question to find novel bugs in real world applications.  To
facilitate this comparison we conducted a study with ZZZ users in which
participants were asked to use the tools in question to hunt for bugs in
popular, real-world applications.  This study revealed that CrashSimulator's
strategy of identifying bugs visible in system call sequences is effective and
usable for developers with diverse backgrounds, across skillsets like software
engineering backgrounds.  Developers with a high degree of proficiency with
operating systems concepts preferred CrashSimulator's ability to target unusual
environmental conditions over manipulating applicaiton data inputs.  Our
participants reported that CrashSimulator's additional setup complexity was
dramatically outweighed by its ability to find new bugs.  This study was
responsible for the identification and reporting of YYY novel bugs, ZZZ of which
have been since been corrected.  With wider use CrashSimulator could help
uncover a great deal of previously unidentified bugs in new and existing
codebases.
\end{abstract}
