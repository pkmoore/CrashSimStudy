\begin{abstract}

A common problem for software developers is applications
exhibiting new bugs \textit{after} deployment.
Preventing bugs related to
network,
operating system,
file system,
and other similar environmental features
is difficult
because there is no way to
know how an application will react to every environment.
Only extensive testing, or failed deployments
can alert a developer to such risks.
Enter
Simulation of Environmental Anomalies (SEA),
a technique that utilizes
evidence of one application's failure
in a given environment
to predict inadequacies in others.
The technique
can extract
unusual properties
found in communications between an application
and its environment.
Simulating these properties
on a running application
exposes it to conditions
that caused failure in other cases.
An incorrect response to the simulation,
is evidence that a bug may exist.
The technique provides a way
to store these anomalies
so,
over time,
applications can be tested
against an increasing set of problematic conditions.
We created a concrete implementation called CrashSimulator
and evaluated it against a set of popular
Linux applications selected
from Coreutils and the Debian popularity contest.
Our tests found a total of 65 bugs in 24 applications
with effects including hangs, crashes, data loss, and remote denial of
service conditions.

\end{abstract}
