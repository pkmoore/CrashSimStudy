\begin{abstract}

A common problem for developers
is applications exhibiting
new bugs \textit{after} deployment.
Many of these bugs
can be traced to
unexpected network,
operating system,
and file system
differences that cause
program executions that were successful in a
development environment to fail once deployed.
Preventing these bugs is difficult because
it is impractical to test an application
in every environment.
Enter Simulating Environmental Anomalies (SEA),
a technique that utilizes evidence
of one application's failure
in a given environment
to generate tests
that can be applied to \textit{other} applications,
to see whether they suffer from analogous faults.
In SEA, models of unusual properties extracted
from interactions between an application, $A$,
and its environment guide simulations
of another application, $B$, running
in the anomalous environment.
This reveals faults $B$
may experience in this environment
without the expense of deployment.
By accumulating these anomalies,
applications can be tested against
an increasing set of problematic conditions.
We implemented a tool called CrashSimulator, which uses SEA,
and evaluated it against Linux applications
selected from Coreutils and the Debian popularity contest.
Our tests found a total of 65 bugs
in 24 applications
with effects including hangs,
crashes,
data loss,
and remote denial of service conditions.

\end{abstract}
