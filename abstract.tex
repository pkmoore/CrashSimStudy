\begin{abstract}

A common problem for software developers is that applications tend to
exhibit new bugs \textit{after} they have been deployed.
Preventing bugs related to
network,
operating system,
file system,
and other similar environmental features
has proven difficult
largely because there is no way to
know how an application will react to every environment.
Up to now,
only time-consuming and expensive testing,
or the painful lesson of a failed deployment
can alert a developer to such risks.
Enter CrashSimulator, a tool that utilizes
evidence of one application's failure
in a given environment
as a predictor of
potential inadequacies
in others.
The key to the tool
is a technique called
\textit{Simulating Environmental Anomalies} (SEA),
which can extract
unusual properties
found in communications between an application
and its environment.
Simulating these properties
on a running application
can expose it to conditions
that caused failure in other cases.
If the application does not respond correctly to a given scenario,
the tool reports that a bug may exist.
These anomalies can then be stored in the tool,
so,
over time,
a developer will be able
to test new applications
against an ever increasing set of problematic conditions.
To test the tool,
we evaluated CrashSimulator against a set of the most popular
Linux applications selected
from the Coreutils project and the Debian popularity contest.
Our tests found a total of 65 bugs in 24 applications,
the consequences of which range from hangs and crashes, to data loss and
remote denial of service conditions.

\end{abstract}
