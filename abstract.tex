\begin{abstract}

It is common that an application that is tested extensively, still has a
large numbers of bugs manifest when deployed in production.  This is
because it is not feasible to test all of an application's potential
\emph{environments}.  These environmental differences, such as slightly
different behavior due to different OS versions, file systems, network
middleboxes, etc. often manifest themselves as failures or crashes of
the application.

CrashSimulator is a technique and tool that uses information about how
different environments differ to identify incorrect application
behavior caused by unhandled anomalous environmental conditions before
the application is deployed.

The tool replays system call traces recorded from the application under
test as a means to monitor and control execution. During replay,
CrashSimulator modifies system call return values and program states to
simulate an anomalous environment.  CrashSimulator makes a determination
about the correctness of the application's behavior as follows: if the
application changes its system call behavior, it might be handling the
anomaly correctly.  If it does not modify its behavior, it is not
responding to the error -- a result indicative of a bug.
This
process allows CrashSimulator to use incorrect behavior identified in
one application running in one environment to determine whether other
applications would make the same mistake in that environment. By
testing an application against a series of these test cases,
CrashSimulator lets the developer see how the application would respond
if executed in a similar environment.  We evaluated CrashSimulator by
using it to test a set of the most popular Linux applications,
including vim, nano, aspell, and others selected from the Coreutils
project and the Debian popularity contest. The result of this
evaluation was 84 bugs identified in 31 applications with consequences
ranging from hangs and crashes to data loss and remote denial of
service conditions.

In order to show that CrashSimulator performs well outside of the lab a
CrashSimulator was compared to two similar tools (AFL and Mutiny) in a
study with ZZZ users.  Participants were asked to use these tools to
hunt for bugs in popular, real-world applications.  This study revealed
that CrashSimulator's strategy of identifying bugs visible in system
call sequences is effective, and that the tool can be used by
developers with diverse backgrounds, and across skill sets.  Though
developers with a high degree of proficiency with operating systems
concepts appreciated CrashSimulator's ability to target unusual
environmental conditions, even subjects had little experience did not
appear to be overwhelmed by the tool's setup and operation.  As YYY
novel bugs were identified and reported during the study, even when
used by novice developers, CrashSimulator is poised to help uncover a
great deal of previously unidentified bugs in new and existing
codebases.

\end{abstract}
