\begin{abstract}

A common problem for software developers is that applications tend to 
exhibit new bugs \textit{after} they have been deployed. 
To this point,
avoiding these bugs has proven difficult largely because there is no way to
know how an application will react to every environment without
time-consuming and expensive testing.
Enter CrashSimulator, a tool that uses information about how environments
differ to identify incorrect behavior \textit{before} the application is
deployed.  Using the key insight that unusual environmental properties can
be encoded as differences present in the communications of an application
with its environment, the tool is able
to simulate anomalous environmental conditions by modifying these
communications and determine if an application is responding to them.

We evaluated CrashSimulator against a set of the most popular
Linux applications selected
from the Coreutils project and the Debian popularity contest.
Our tests found a total of XXXX bugs in YYYY applications,
the consequences of which range from hangs and crashes, to data loss and
remote denial of service conditions.  In addition, we conducted
a user study in which
ZZZ participants tested real world applicaitons using CrashSimulator.
This effort resulted in the identification of an additional WWWW new
bugs.
Our combined results demonstrate
that CrashSimulator's strategy of identifying bugs
is effective, and yielded results for
developers with diverse backgrounds, and across skill sets.

\end{abstract}
