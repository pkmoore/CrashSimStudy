\section{Background}
\label{SEC:background}

\subsection{Criteria}

\preston{Not surehow to introduce this sections so I'm just going straight
into it until I get a better idea.} CrashSimulator's system call based
technique its user to replicate anomalous environmental conditions in a
wide variety of operating systems domains.  In future sections this paper
how we used CrashSimulator to test applications' responses to environmental
anomalies that could be expressed through file system and network related
system calls.  We decided to compare CrashSimulator against two existing
testing tools, one that is file based (AFL) and another that performs
network oriented testing.

\subsection{CrashSimulator}

CrashSimulator finds bugs in applications by exposing them to weird
environmental conditions that developers may not have anticipated.  This is
accomplished by encoding the environmental conditions as a set of
modifications made to the results an side effects of system calls made by
the application.  CrashSimulator injects these anomalies by recording,
modifying, and replaying the results and side effects of the system calls
made by the application.  This process allows experts to construct "tests"
that developers can use to evaluate their applications without their having
to understand the operating system complexities underlying the
environmental conditions simulated by the test. This approach has proven
successful in finding new bugs in many widely deployed applications.



\subsection{AFL}

AFL (American Fuzzy Lop) is a file-based fuzzer that tests an application
by mutating the contents of a candidate file provided by the user.  It
offers two improvements the naive approach of simply passing random data to
an application through a file.  First, AFL bases its mutations on a valid
file for the application.  This increases the chance that the mutated file
will make deeper into processing code improving coverage.  Second, AFL
provides a set of C compiler customizations that add instrumentation code
to the application being tested at compile time.  This instrumentation
allows AFL to tune its mutations to more thoroughly exercise the
application's code paths.  The intention when testing with AFL is to
generate an input file that makes the application crash.  Once a
crash-causing file is discovered, users must manually analyze the file and
application in order to ascertain the cause of the crash and fix it.
Though there is some support for testing closed source applications through
a hypervisor AFL, is intended to test applications written in C on Linux
and has an impressive track record in terms of bugs discovered in major
applications that meet this criteria.


\subsection{Mutiny Fuzzer}

At a high level, mutiny has an architecture similar to CrashSimulator

Mutiny operates on recordings of valid network communication.

Mutiny mutates these recordings using various strategies provided by
mutator plugins.

Mutiny replays this traffic and monitors how the application responds to
the injected mutations.
