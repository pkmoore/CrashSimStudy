\section{Background}
\label{SEC:background}

\subsection{Criteria}

\preston{Not surehow to introduce this sections so I'm just going straight
into it until I get a better idea.} CrashSimulator's system call based
technique its user to replicate anomalous environmental conditions in a
wide variety of operating systems domains.  In future sections this paper
how we used CrashSimulator to test applications' responses to environmental
anomalies that could be expressed through file system and network related
system calls.  We decided to compare CrashSimulator against two existing
testing tools, one that is file based (AFL) and another that performs
network oriented testing.

\subsection{CrashSimulator}

CrashSimulator finds bugs in applications by exposing them to weird
environmental conditions that developers may not have anticipated.  This is
accomplished by encoding the environmental conditions as a set of
modifications made to the results an side effects of system calls made by
the application.

CrashSimulator injects these anomalies by recording, modifying, and
replaying the results and side effects of the system calls made by the
application.

This process allows experts to construct "tests" that developers can use to
evaluate their applications without their having to understand the
operating system complexities underlying the environmental conditions
simulated by the test.

CrashSimulator uses Mozilla's rr record-and-replay debugger to handle
complex replay situations.

CrashSimulator's supervisor attaches to the processes being replayed at
strategic points in time in order to manipulate system call behavior.

CrashSimulator observes whether or not the application responds to the
modified system call behavior and classifies the application's behavior as
maybe correct or definitely wrong.

CrashSimulator is able to test closed source applications.

CrashSimulator was able to find many new bugs in widely deployed
applications.


\subsection{AFL}

AFL (American Fuzzy Lop) is a file-based fuzzer that tests an application
by mutating the contents of a candidate file provided by the user.

AFL improves on similar tools by providing a set of C compiler
customizations that add instrumentation code to the application being
tested at compile time.  This instrumentation allows AFL to tune its
mutations to more thoroughly exercise the application's code paths.

AFL's goal is to make the application crash, most commonly by triggering an
application state where memory is corrupted.

AFL is intended for testing application written in C on Linux.

AFL offers an experimental mode for testing closed source applications by
running them under a virtual machine hypervisor.

AFL is responsible for the discovery of many novel bugs in major open
source applications.

\subsection{Mutiny Fuzzer}

At a high level, mutiny has an architecture similar to CrashSimulator

Mutiny operates on recordings of valid network communication.

Mutiny mutates these recordings using various strategies provided by
mutator plugins.

Mutiny replays this traffic and monitors how the application responds to
the injected mutations.
