\section{Conclusion}
\label{SEC:conclusion}

As we have discussed,
it is common for an application 
to fail upon deployment because of unexpected interactions
with its environment.
Although finding and eliminating
faults in an application is a key concern for software developers, it is
impractical test it in every environment it will face.
To address this problem, we developed the SEA
technique.
SEA is beneficial for developers because it allows
the effort spent debugging failures in a given environment
to be preserved and reused programmatically to test whether
future applications will also fail.
As this process is repeated,
a corpus of bug-causing aspects,
known as ``anomalies'' can be accumulated
resulting in an ever-increasing capability
to test applications in situations
that proved problematic in the past.

We built a concrete implementation of SEA
called CrashSimulator which implements
the technique by simulating environmental
anomalies visible in the system calls an application makes.
Operating on system calls gives the tool a ``universal'' way to
encode and inject anomalies. Consequently, a set of mutations can be
collected from existing applications for use in testing others.
In this way, an ever-expanding corpus of anomalies can be
created, allowing lessons learned from bugs in one application to benefit
many others.
Our evaluation of CrashSimulator
has shown that this technique works and is
effective at finding bugs in well tested software.
In total,
XXXX new bugs were identified in popular applications.
These bugs, if triggered in the wild,
could lead to effects ranging from simple program hangs
to security vulnerabilities and data loss.
Of these, YYYY been reported to the
affected parties.

Given that the technique as proven sound
and our has proven effective we will continue to improve upon it.
We envision a public repository of anomalies
that can be applied to new or existing applications.
We are also exploring
opportunities to further automate the discovery process
and improve the way anomalies are specified using a
domain specific language.
Further future work
will focus on analyzing how an
application attempts
to recover from the anomalies.  This would allow
us to determine whether
an application is correctly recovering
from an error, or carrying out some incorrect response.
