\section{Conclusion}
\label{SEC:conclusion}

As our first round of tests proved,
applications often fail upon deployment because of unexpected interactions
with its environment.
Although finding and eliminating
faults in applications is a key concern for software developers, it is
impractical to do actual tests in every possible
environment it may face.
To address this problem, we developed CrashSimulator
in order to determine whether an application will
respond correctly to unexpected conditions.

CrashSimulator's
technique of
operating on system calls gives the tool a ``universal'' way to
encode and inject anomalies. Consequently, a set of mutations can be
collected from existing applications for use in testing others.
In this way, an ever-expanding ``test suite'' can be
created, allowing lessons learned from bugs in one application to benefit
many others.
Our two-stage evaluation of CrashSimulator
has shown it to be both usable and
effective.  It received praise in our participant surveys and,
while it is particularly powerful in the hands of developers with more
experience,
even those with little experience were pleased that
the expert knowledge encoded in the tool
let them find bugs.
In our user study, a total of
XXXX new bugs were identified in popular applications.
These bugs have been reported to the
affected parties and YYYY have already been corrected.

In the long term, we
envision this testing strategy will yield a public repository of anomalies
that can be applied to new or existing applications.
We are also exploring
opportunities to further automate the discovery process.
Beyond these logical
evolutions of the tool, our future work
will focus on analyzing how an
application attempts
to recover from the anomalies.  This would allow
us to determine whether
an application is correctly recovering
from an error, or carrying out some incorrect response.
Second, we will work on making it possible for CrashSimulator to
understand and mutate common data formats, allowing it to test using
anomalies that appear within the data returned by system calls like {\tt
read()} and {\tt recv()}.
