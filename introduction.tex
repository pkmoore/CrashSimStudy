\section{Introduction}
\label{SEC:introduction}

\preston{Argue that you have to have automated testing}
Automated testing is a necessity of developing complex, modern, software
systems.  There are simply too many moving parts to orchestrate and test
manually.  Industry experience has borne this out --- BBB percent of
projects hosted on GitHub employ some sort of automated testing of the code
checked into their repositories and most won't allow code to be checked in
until it has passed some testing regime.\preston{cite}

Care must be taken in designing and carrying out this testing as there a
real chance of introducing new inefficiencies and pain points into a
projects development life cycle.  To this end, a few points are worthy of
consideration. First, reliance on experts and human-in-the-loop strategies
doesn't scale.\preston{cite Alvaro}  There simply aren't enough experts to
go around and a developer's time is far better spent on actual development
than manual testing.\preston{cite developer shortages}  Fortunately, well
constructed automated testing tools allow two birds to be killed with one
stone -- we can spare developer time for more creative tasks and we can
take advantage of the expert knowledge encoded in the tools to temper the
lack our shortage of human experts.

\preston{Discuss what properties an automated testing tool must have}

Second, simply grabbing a tool of the shelf (or internet) won't do.  It
must capable of testing the system in question, it must be usable by the
developers that on staff, and it must produce high-quality results.

\preston{Talking about source code here feels out of place but I think it
needs to go somewhere}
One of the major limitations of many automated testing tools is a reliance
on the availability of a target application's source code.  Out of the top
XXX automated testing tools (as ranked by YYYY), ZZZZ have this
dependency.\preston{cite} While open source software has dramatically
increased in popularity over recent years, closed source dependencies are
still a major concern.\preston{cite}  Any tool that requires access to
source code will have diminished capability in projects incorporating
closed source components.

In addition to a general shortage of development capacity, another limiting
factor can be the developer's skill sets and expertise.  In order to be
successful, a testing tool must be usable by developers working on a
system.  If the developers lack the background necessary to deal with the
output of a tool then the effort surrounding the tool has been wasted.

Even if a tool satisfies the above requirements, its usefulness is
dependant on the quality of its output.  Tools that produce a high number
of false positives are quickly deemed untrustworthy.  A study by ZZZZ as
shown that false positive rates as low as Q\% have a dramatic, negative
impact on developer perceptions of a tool.  \preston{cite}  A similar
finding is supported by current psychological literature.  Individuals tend
to over-emphasize and over-predict statistically rare negative
events.\preston{cite}  Put another way, if a tool leads to a developer
wasting his or her afternoon by reporting a bug that doesn't exist, the
developer is going to mistrust the tool from then on out.

A useful tool must also do a good job of localizing the source of a bug.
Tools that fail to identify (or worse, misidentify) the source of bugs with
sufficient detail can also lead to wasted time and effort.  For example, a
fuzzing tool that only interacts with an application by mutating normal
input may eventually uncover a crash but merely reporting the presence of a
crash-causing input doesn't get the developer very close to identifying and
fixing the responsible block of code.

\preston{This argument feels weak to me}
At the same time, output that is too detailed or low level can prove just
as detrimental to the bug fixing process.  Consider the case where a tool
reports a bug takes place during the execution of code inside a library
compiled without debugging symbols.  If the developer addressing the bug is
unfamiliar with working on disassembled code, identifying the cause of the
bug may be difficult.

\preston{Talk about how CrashSimulator has these properties}
In this paper we want to show that CrashSimulator is able to assist
developers of varying skill levels in identifying bugs bugs in real world
applications.  Additionally, we show that CrashSimulator provides value in
mapping the presence of these bugs to the units of code responsibile for
them allowing developers to fix them in a more timely manner.

CrashSimulator's effectiveness comes its ability expose applications to
unusual environmental conditions that can cause problems with an
application's execution.  CrashSimulator's built-in set of anomalous
conditions allows its users to define which conditions to test an
application against allieviating the need for the expert knowledge required
to set up and test an application in the chosen environments.

CrashSimulator also allows application misbehavior to be localized to a
sequence of system calls allowing the true source of a bug to be more
quickly identified easing the process of correcting it.  This is an
advantage over tools that simply indicate the presence of a bug (i.e.  this
input caused a crash)

\preston{Talk about we prove CrashSimulator has those properties}
To validate these claims we conducted a study with ZZZ participants
consisting of Master's computer science students with varying backgrounds
and specializations.  We asked these participants to test existing popular
applications (as ranked by Debian's Popularity Contest) using each of the
tools.  From this work we collected both quantitative results in terms of
numbers of bugs identified and qualitative results about the tools' user
experince through surveys.

The main contributions in this work are as follows:
\preston{Item 3 needs to be reworded}
\begin{enumerate}

\item We illustrate that CrashSimulator is useful and usable by developers
with a wide variety of backgrounds.

\item We show that CrashSimulator allows these developers to find real bugs
in real applications.

\item We demonstrate that CrashSimulator compares favorably against similar
automated testing tools in the number of bugs found and in user opinion on
usefulness of the tools' output.

\end{enumerate}
